\documentclass[11pt]{report} 
\begin{document}
%%%%%%%%%%%%%%%%%%%%%%%%%%%%%%%%%%%%%%%%%%%%%%%%%%%%%%%%%%%%%%%%%
%%%%%%%%%%%%%%%%%%%%%%%%%%%%%%%%%%%%%%%%%%%%%%%%%%%%%%%%%%%%%%%%%
\section{Wide-field imaging, deconvolution ({\tt widefield})}
\label{section:im.widefield}

When imaging sufficiently large angular regions, the sky can no longer
be treated as a two-dimensional plane and the use of the {\tt clean}
task will produce bright source distortions that become
increasingly large with distance from the phase center.  The task
{\tt widefield} should then be used.  It contains most, but not all,
of the flexibility of {\tt clean}.  Besides making large area
undistorted images, {\tt widefield} must be used to image isolated
regions (i.e. confusing sources) is they are distant from the field
center.

When is {\tt widefield} needed?  It depends on the expected dynamic
range the image.  In order to keep the phase error associated with the
sky/array curvature less than about $5^\circ$ (good to about 500:1
dynamic range), use the following table, suitably scaled, for
guidance:

\scriptsize
\begin{verbatim}
        Maximum Radius of Image Before widefield is Needed
          Assuming 5 deg phase error and 35-km Baseline

           Wavelength                 Radius of image
              6 cm                       1.4 arcmin
             20 cm                       2.6 arcmin
             90 cm                       5.3 arcmin

  Radius of image ~ SQRT (Wavelength * phase error / Maximum baseline)
      (arcmin)              (cm)          (deg)           (km)
\end{verbatim}
\normalsize

\noindent If a relatively small image is being made, but there are outliers
sources beyond the above limits, then widefield should also be used.

The task {\tt widefield} can be used in two major modes: First, the
wf-projection method deals with the w-term (the phase associated with
the sky/array curvature) internally.  Secondly, the image can be
broken into many facets, each small enough so that the w-term is not
significant.  These two basic methods can be combined, and are discussed
below.

\medskip

The default inputs to {\tt widefield} are:
\scriptsize
\begin{verbatim}
#  widefield :: Wide-field imaging and deconvolution with selected algorithm
vis                 =       ['']        #  name of input visibility file
imagename           =                   #  Pre-name of output images
outlierfile         =         ''        #  Text file with image names, sizes, centers
field               =         ''        #  Field Name
spw                 =         ''        #  Spectral windows:channels: '' is all
selectdata          =      False        #  Other data selection parameters
mode                =      'mfs'        #  Type of selection (mfs, channel, velocity, frequency)
niter               =        500        #  Maximum number of iterations
gain                =        0.1        #  Loop gain for cleaning
threshold           =    '0.0Jy'        #  Flux level to stop cleaning.  Must include units
psfmode             =    'clark'        #  Algorithm to use (clark, hogbom)
ftmachine           = 'wproject'        #  Gridding method for the image (wproject, ft)
     wprojplanes    =        256        #  Number of planes to use in wprojection convolutiuon function
     facets         =          1        #  Number of facets along each axis in main image only

multiscale          =         []        #  set deconvolution scales (pixels), default: multiscale=[]
                                        #   (standard CLEAN)
interactive         =      False        #  use interactive clean (with GUI viewer)
mask                =         []        #  cleanbox(es), mask image(s), and/or region(s)
imsize              = [256, 256]        #  Image size in pixels (nx,ny), symmetric for single value
cell                = ['1.0arcsec', '1.0arcsec'] #  The image cell size in arcseconds [x,y].
phasecenter         =         ''        #  Field Identififier or direction of the image phase center
restfreq            =         ''        #  rest frequency to assign to image (see help)
stokes              =        'I'        #  Stokes params to image (I,IV,QU,IQUV,RR,LL,XX,YY,RRLL,XXYY)
weighting           =  'natural'        #  Weighting to apply to visibilities
cyclefactor         =        1.5        #  Change the threshold at which the deconvolution cycle will stop,
                                        #   degrid, and subtract from the visiblities.
cyclespeedup        =         -1        #  Cycle threshold doubles in this number of iterations
uvtaper             =      False        #  Apply additional uv tapering of  visibilities.
restoringbeam       =       ['']        #  Output Gaussian restoring beam for CLEAN image
async               =      False        #  If true the taskname must be started using widefield(...)
\end{verbatim}
\normalsize

\noindent Most of the parameters are similar to {\tt clean} and should be reviewed
there.  Specialized parameters for {\tt widefield} are as follows:

The parameters {\tt outlierfield} versus {\tt imsize} + {\tt
phasecenter}: When using widefield, the position and imagesize of each
image must be specified.  The following example shows two methods for
specifying a big image with four outliers:

\scriptsize
\begin {verbatim}
vis                 = 'wfield.ms'       #  name of input visibility file
imagename           = 'wf'              #  Pre-name of output images
outlierfile         = 'setup.txt'       #  Text file with image names, sizes, centers
imsize              = [256, 256]        #  Image size in pixels (nx,ny), symmetric for single value
cell                = '1.0arcsec'       #  The image cell size in arcseconds [x,y].
phasecenter         =         ''        #  Field Identififier or direction of the image phase center

The file "setup.txt" is:

C   main 2048 2048   13 27 20.98     43 26 28.0   # Main field with image size and phase center
C   out1  128  128   13 30 52.158    43 23 08.00  # First outlier field specification
                      etc
The C in column 1 must be present, although it not presently used.
\end{verbatim}
\normalsize

\noindent The {\tt cell} is identical for all fields, and {\tt imsize} and {\tt
phasecenter} are not used.  The image pre-names would be: wf.main,
wf.out1, etc

The alternative input, needing no additional outlier file, is: 

\scriptsize
\begin{verbatim}
vis                 = 'wfield.ms'       #  name of input visibility file
imagename           = 'wf'              #  Pre-name of output images
outlierfile         = 'image_setup.txt' #  Text file with image names, sizes, centers
imsize              = [2048, 2048, 128, 128] #  Image size in pixels (nx,ny)
cell                = '1.0arcsec'       #  The image cell size in arcseconds [x,y].
phasecenter         = ['J2000 13h27m20.98 43d26m28.0', 'J2000 13h30m52.158 43d23m08.00']
\end{verbatim}
\normalsize

The crucial part of {\tt widefield} is the parameters under {\tt ftmachine}.  The three
types of use are: (1) w-projection only; (2) facets only; (3) w-projection with facets.

(1) Pure wprojection method

\scriptsize
\begin{verbatim}
ftmachine           = 'wproject'        #  Gridding method for the image (wproject, ft)
     wprojplanes    =         64        #  Number of planes to use in wprojection convolutiuon function
     facets         =          1        #  Number of facets along each axis in main image only
\end{verbatim}

\noindent A reasonable value is wprojplanes = Bmax(k$\lambda)\times$
imagewidth($arcmin^2) / 600$), with a minimum value of 16.  The
wprojection algorithm is much faster than using faceting, but it does
consume a lot of memory.  On most 32-bit machines with 1 or 2 Mbytes
of memory, images larger than about $4000\times 4000$ cannot be made.

(2) Faceting only (breaking image into many small parts)

\scriptsize
\begin{verbatim}
ftmachine           = 'ft'        #  Gridding method for the image (wproject, ft)
     facets         =          7        #  Number of facets along each axis in main image only
                                           i.e. in this case 7x7 facets
\end{verbatim}
\normalsize

\noindent A reasonable value of facets is such that the imagewidth of each facet does not need
the w-term correction.  The computation method with pure faceting is slow, so that wprojection
is recommended


(3) Combination of wprojection and faceting:

\scriptsize
\begin{verbatim}

ftmachine           = 'wproject'        #  Gridding method for the image (wproject, ft)
     wprojplanes    =         32        #  Number of planes to use in wprojection convolutiuon function
     facets         =          3        #  Number of facets along each axis in main image only

\end{verbatim}
\normalsize

\noindent This hybrid method allows for a smaller wprojplanes in order
to try to conserve memory if the image size approached the memory limit
of the computer.  However, there is a large penalty in execution time.

The other parameters are described in {\tt clean}, and multi-scale
cleaning in available.  Mosaicing and MEM is not yet included in {\tt
widefield}.



%%%%%%%%%%%%%%%%%%%%%%%%%%%%%%%%%%%%%%%%%%%%%%%%%%%%%%%%%%%%%%%%%
%%%%%%%%%%%%%%%%%%%%%%%%%%%%%%%%%%%%%%%%%%%%%%%%%%%%%%%%%%%%%%%%%
\end{document}
