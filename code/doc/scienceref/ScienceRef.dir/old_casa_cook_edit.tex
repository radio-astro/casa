%%%%%%%%%%%%%%%%%%%%%%%%%%%%%%%%%%%%%%%%%%%%%%%%%%%%%%%%%%%%%%%
%%%%%%%%%%%%%%%%%%%%%%%%%%%%%%%%%%%%%%%%%%%%%%%%%%%%%%%%%%%%%%%%%
%%%%%%%%%%%%%%%%%%%%%%%%%%%%%%%%%%%%%%%%%%%%%%%%%%%%%%%%%%%%%%%%%

% STM 2007-04-13  split from previous version
% STM 2007-06-25  bring up to Alpha Patch 1 level
% MPR 2007-07-05  minor tweaks, esp. to match frozen alpha-1 release of 4jul07
% STM 2007-09-20  pre-beta version
% STM 2007-10-09  Gustaaf's corrections
% STM 2007-10-09  beta version (spell-checked)
% STM 2007-11-09  beta 0.5, qcasabrowser
% STM 2008-03-24  patch 1.0
% STM 2008-05-14  start patch 2.0
% STM 2008-09-30  Patch 3 editing start, extendflag
% STM 2009-05-19  Patch 4 editing start, plotants and plotms
% STM 2009-11-24  Release 3.0.0 editing start

\chapter{Data Examination and Editing}
\label{chapter:edit}

%%%%%%%%%%%%%%%%%%%%%%%%%%%%%%%%%%%%%%%%%%%%%%%%%%%%%%%%%%%%%%%%%
%%%%%%%%%%%%%%%%%%%%%%%%%%%%%%%%%%%%%%%%%%%%%%%%%%%%%%%%%%%%%%%%%
\section{Plotting and Flagging Visibility Data in CASA}
\label{section:edit.intro}

The tasks available for plotting and flagging of data are:
\begin{itemize}
   \item {\tt flagmanager} --- manage versions of data flags
      (\S~\ref{section:edit.flagmanager})
   \item {\tt flagautocorr} --- non-interactive flagging of auto-correlations
      (\S~\ref{section:edit.flagautocorr})
   \item {\tt plotms} --- prototype next-generation X-Y MS plotter
      (\S~\ref{section:edit.plot.plotms})
   \item {\tt plotxy} --- create X-Y plots of data in MS, flag data
      (\S~\ref{section:edit.plot.plotxy})
   \item {\tt flagdata} --- non-interactive flagging of data
      (\S~\ref{section:edit.flagdata})
   \item {\tt browsetable} --- browse data in any CASA table (including a MS)
      (\S~\ref{section:edit.browse})
   \item {\tt plotants} --- create simple plots of antenna positions
      (\S~\ref{section:edit.plot.plotants})
\end{itemize}

The following sections describe the use of these tasks.

Information on other related operations can be found in:
\begin{itemize}
   \item {\tt listobs} --- list summary of a MS (\S~\ref{section:io.list})
   \item {\tt listvis} --- list data in a MS (\S~\ref{section:io.vis.listvis})
   \item {\tt selectdata} --- general data selection syntax
      (\S~\ref{section:io.selection})
   \item {\tt viewer} --- use the {\tt casaviewer} to display the MS as a 
      raster image, and flag it (\S~\ref{chapter:display})

\end{itemize}

%%%%%%%%%%%%%%%%%%%%%%%%%%%%%%%%%%%%%%%%%%%%%%%%%%%%%%%%%%%%%%%%%
%%%%%%%%%%%%%%%%%%%%%%%%%%%%%%%%%%%%%%%%%%%%%%%%%%%%%%%%%%%%%%%%%
\section{Managing flag versions with {\tt flagmanager}}
\label{section:edit.flagmanager}

The {\tt flagmanager} task will allow you to manage different
versions of flags in your data.  These are stored inside a CASA
flagversions table, under the name of the MS {\tt <msname>.flagversions}.
For example, for the MS {\tt jupiter6cm.usecase.ms}, there will need
to be {\tt jupiter6cm.usecase.ms.flagversions} on disk.  This is
created on import (by {\tt importvla} or {\tt importuvfits}) or
when flagging is first done on an MS without a {\tt .flagversions}
(e.g. with {\tt plotxy}).  

By default, when the {\tt .flagversions} is created, this
directory will contain a {\tt flags.Original} in it containing
a copy of the original flags in the {\tt MAIN} table of the MS
so that you have a backup.  It will also contain a file called
{\tt FLAG\_VERSION\_LIST} that has the information on the various
flag versions there.

The inputs for {\tt flagmanager} are:
\small
\begin{verbatim}
vis                 =         ''        #   Name of input visibility file (MS)
mode                =     'list'        #   Flag management operation (list,save,restore,delete)
\end{verbatim}
\normalsize

The {\tt mode='list'} option will list the available flagversions from
the {\tt <msname>.flagversions} file.  For example:
\small
\begin{verbatim}
CASA <102>: default('flagmanager')
CASA <103>: vis = 'jupiter6cm.usecase.ms'
CASA <104>: mode = 'list'
CASA <105>: flagmanager()
MS : /home/imager-b/smyers/Oct07/jupiter6cm.usecase.ms

main : working copy in main table
Original : Original flags at import into CASA
flagautocorr : flagged autocorr
xyflags : Plotxy flags
\end{verbatim}
\normalsize

The {\tt mode} parameter expands the options.  For example, if you wish to 
save the current flagging state of {\tt vis=<msname>}, 
\small
\begin{verbatim}
mode                =     'save'        #   Flag management operation (list,save,restore,delete)
     versionname    =         ''        #   Name of flag version (no spaces)
     comment        =         ''        #   Short description of flag version
     merge          =  'replace'        #   Merge option (replace, and, or)
\end{verbatim}
\normalsize
with the output version name specified by {\tt versionname}.  For example, the
above {\tt xyflags} version was written using:
\small
\begin{verbatim}
   default('flagmanager')
   vis = 'jupiter6cm.usecase.ms'
   mode = 'save'
   versionname = 'xyflags'
   comment = 'Plotxy flags'
   flagmanager()
\end{verbatim}
\normalsize
and you can see that there is now a sub-table in the flagversions directory
\small
\begin{verbatim}
CASA <106>: ls jupiter6cm.usecase.ms.flagversions/
  IPython system call: ls -F jupiter6cm.usecase.ms.flagversions/
  flags.flagautocorr  flags.Original  flags.xyflags  FLAG_VERSION_LIST
\end{verbatim}
\normalsize
{\bf It is recommended that you use this facility regularly to save versions
during flagging.}

You can restore a previously saved set of flags using the 
{\tt mode='restore'} option:
\small
\begin{verbatim}
mode                =  'restore'        #   Flag management operation (list,save,restore,delete)
     versionname    =         ''        #   Name of flag version (no spaces)
     merge          =  'replace'        #   Merge option (replace, and, or)
\end{verbatim}
\normalsize
The {\tt merge} sub-parameter will control the action.  
For {\tt merge='replace'}, the flags in {\tt versionname} will replace those
in the MAIN table of the MS.  For {\tt merge='and'}, only data that is
flagged in BOTH the current MAIN table and in {\tt versionname} will 
be flagged.  For {\tt merge='or'}, data flagged in EITHER the MAIN or
in {\tt versionname} will be flagged.

The {\tt mode='delete'} option can be used to remove {\tt versionname} from the
flagversions:
\small
\begin{verbatim}
mode                =   'delete'        #   Flag management operation (list,save,restore,delete)
     versionname     =         ''       #   Name of flag version (no spaces)
\end{verbatim}
\normalsize

 
%%%%%%%%%%%%%%%%%%%%%%%%%%%%%%%%%%%%%%%%%%%%%%%%%%%%%%%%%%%%%%%%%
%%%%%%%%%%%%%%%%%%%%%%%%%%%%%%%%%%%%%%%%%%%%%%%%%%%%%%%%%%%%%%%%%
\section{Flagging auto-correlations with {\tt flagautocorr}}
\label{section:edit.flagautocorr}

The {\tt flagautocorr} task can be used if all you want to do
is to flag the auto-correlations out of the MS.  Nominally,
this can be done upon filling from the VLA for example, but
you may be working from a dataset that still has them.

This task has a single input, the MS file name:
\small
\begin{verbatim}
vis                 =         ''        #   Name of input visibility file (MS)
\end{verbatim}
\normalsize
To use it, just set and go:
\small
\begin{verbatim}
CASA <90>: vis = 'jupiter6cm.usecase.ms'                                         
CASA <91>: flagautocorr()
\end{verbatim}
\normalsize

Note that the auto-correlations can also be flagged using 
{\tt flagdata} (\S~\ref{section:edit.flagdata}) but the 
{\tt flagautocorr} task is an handy shortcut for this common
operation.

%%%%%%%%%%%%%%%%%%%%%%%%%%%%%%%%%%%%%%%%%%%%%%%%%%%%%%%%%%%%%%%%%
%%%%%%%%%%%%%%%%%%%%%%%%%%%%%%%%%%%%%%%%%%%%%%%%%%%%%%%%%%%%%%%%%
\section{X-Y Plotting and Editing of the Data}
\label{section:edit.plot}

There are three main X-Y plotting tasks in CASA:
\begin{itemize}
   \item {\tt plotms} --- prototype next-generation X-Y MS plotter
      (\S~\ref{section:edit.plot.plotms})
   \item {\tt plotxy} --- create X-Y plots of data in MS, flag data
      (\S~\ref{section:edit.plot.plotxy})
   \item {\tt plotants} --- create simple plots of antenna positions
      (\S~\ref{section:edit.plot.plotants})
\end{itemize}

%%%%%%%%%%%%%%%%%%%%%%%%%%%%%%%%%%%%%%%%%%%%%%%%%%%%%%%%%%%%%%%%%
\subsection{MS Plotting and Editing using {\tt casaplotms}}
\label{section:edit.plot.plotms}

{\bf ALERT:}
The {\tt casaplotms} tool is a standalone
plotter application based on Qt.  It is intended to replace
{\tt plotxy} and will be more fully functional in the next release.

You can also start the application outside
CASA (e.g.\ by typing {\tt casaplotms}) or from inside the shell
using the OS command syntax, either {\tt !casaplotms} from IPython,
or {os.system('casaplotms')} from IPython or script.

The current inputs to the {\tt plotms} task are:
\small
\begin{verbatim}
#  plotms :: A plotter/interactive flagger for visibility data.
vis            =      ''   #  input visibility dataset (blank for none)
xaxis          =      ''   #  plot x-axis (blank for default/current)
yaxis          =      ''   #  plot y-axis (blank for default/current)
selectdata     =    True   #  data selection parameters
   field       =      ''   #  field names or field index numbers (blank for all)
   spw         =      ''   #  spectral windows:channels (blank for all)
   timerange   =      ''   #  time range (blank for all)
   uvrange     =      ''   #  uv range (blank for all)
   antenna     =      ''   #  antenna/baselines (blank for all)
   scan        =      ''   #  scan numbers (blank for all)
   correlation =      ''   #  correlations (blank for all)
   array       =      ''   #  (sub)array numbers (blank for all)
   msselect    =      ''   #  MS selection (blank for all)

averagedata    =    True   #  data averaging parameters
   avgchannel  =      ''   #  average over channel?  (blank = False, otherwise value in channels)
   avgtime     =      ''   #  average over time? (blank = False, other value in seconds)
   avgscan     =   False   #  only valid if time averaging is turned on.  average over scans?
   avgfield    =   False   #  only valid if time averaging is turned on.  average over fields?
   avgbaseline =   False   #  average over all baselines?  (mutually exclusive with avgantenna)
   avgantenna  =   False   #  average by per-antenna?  (mutually exclusive with avgbaseline)
   avgspw      =   False   #  average over all spectral windows?

extendflag     =   False   #  have flagging extend to other data points?
plotxycomp     =   False   #  use plotxy compliance parameters?
async          =   False   #  If true the taskname must be started using plotms(...)
\end{verbatim}
\normalsize

You can find a short example of using {\tt casaplotms} on the
CASAGuides Wiki:
\begin{quote}
   \url{http://casaguides.nrao.edu/index.php?title=Data_flagging_with_casaplotms}
\end{quote}

%%%%%%%%%%%%%%%%%%%%%%%%%%%%%%%%%%%%%%%%%%%%%%%%%%%%%%%%%%%%%%%%%
\subsection{Plotting and Editing using {\tt plotxy}}
\label{section:edit.plot.plotxy}

{\bf ALERT:} The {\tt plotxy} code is fragile and slow, and is being
replaced by the {\tt plotms} (\S~\ref{section:edit.plot.plotms}).
We retain {\tt plotxy} in this release as not all functionality is
available yet in {\tt plotms}.

\begin{wrapfigure}{r}{2.5in}
  \begin{boxedminipage}{2.5in}
     \centerline{\bf Inside the Toolkit:}
     Access to {\tt matplotlib} is also provided through 
     the {\tt pl} tool. 
     See below for a description of the {\tt pl} tool functions. 
  \end{boxedminipage}
\end{wrapfigure}

The principal way to get X-Y plots of visibility data is using the
{\tt plotxy} task.  This task also provides editing capability.
CASA uses the {\tt matplotlib} plotting library to display its plots.
You can find information on {\tt matplotlib} at
\url{http://matplotlib.sourceforge.net/}.

\begin{figure}[h!]
\begin{center}
%\gname{uvcoverage}{5.5}
\pngname{plotxy_jupiter}{5.5}
\caption{\label{fig:matplotlib}The {\tt plotxy} plotter, showing the
  Jupiter data versus uv-distance.  You can see bad data in this plot.
  The {\bf bottom set of buttons} on the
  lower left are: 1,2,3) {\bf Home, Back, and Forward}. Click to
  navigate between previously defined views (akin to web navigation).
  4) {\bf Pan}. Click and drag to pan to a new position. 5) {\bf
  Zoom}. Click to define a rectangular region for zooming. 6) {\bf
  Subplot Configuration}. Click to configure the parameters of the
  subplot and spaces for the figures. 7) {\bf Save}. Click to launch a
  file save dialog box.  The {\bf upper set of buttons in the lower left} are:
  1) {\bf Mark Region}. Press this to begin marking regions (rather than
  zooming or panning).  2,3,4) {\bf Flag, Unflag, Locate}.  Click on these
  to flag, unflag, or list the data within the marked regions.  5) {\bf Next}.
  Click to move to the next in a series of iterated plots.
  Finally, the {\bf cursor readout} is on the bottom right.}
\hrulefill
\end{center}
\end{figure}

To bring up this plotter use the {\tt plotxy} task.  The inputs are: 
\small
\begin{verbatim}
#  plotxy :: X-Y plotter/interactive flagger for visibility data

vis              =         ''   #  Name of input visibility
xaxis            =     'time'   #  X-axis: def = 'time': see help for options
yaxis            =      'amp'   #  Y-axis: def = 'amp': see help for options
     datacolumn  =     'data'   #  data (raw), corrected, model, residual (corrected - model)

selectdata       =      False   #  Other data selection parameters
spw              =         ''   #  spectral window:channels: ''==>all, spw='1:5~57'
field            =         ''   #  field names or index of calibrators: ''==>all
averagemode      =         ''   #  Select averaging type: 'vector', 'scalar'
restfreq         =         ''   #  a frequency quanta or transition name. see help for options
extendflag       =      False   #  Have flagging extend to other data points?
subplot          =        111   #  Panel number on display screen (yxn)
plotsymbol       =        '.'   #  Options include . : , o ^ v > < s + x D d 2 3 4 h H | _
plotcolor        =  'darkcyn'   #  Plot color
plotrange        = [-1, -1, -1, -1]  #  The range of data to be plotted (see help)
multicolor       =     'corr'   #  Plot in different colors: Options: none, both, chan, corr
selectplot       =      False   #  Select additional plotting options (e.g, fontsize, title,etc)
overplot         =      False   #  Overplot on current plot (if possible)
showflags        =      False   #  Show flagged data?
interactive      =       True   #  Show plot on gui?
figfile          =         ''   #  ''= no plot hardcopy, otherwise supply name
async            =      False   #  If true the taskname must be started using plotxy(...)
\end{verbatim}
\normalsize

{\bf ALERT:} The {\tt plotxy} task expects all of the scratch columns to
be present in the MS, even if it is not asked to plot the contents.
If you get an error to the effect {\tt "Invalid Table operation:
Table: cannot add a column"} then use {\tt clearcal()} to force these
columns to be made in the MS.  Note that this will clear anything in
all scratch columns (in case some were actually there and being used).

Setting {\tt selectdata=True} opens up the selection sub-parameters:
\small
\begin{verbatim}
selectdata       =       True   #   Other data selection parameters
     antenna     =         ''   #   antenna/baselines: ''==>all, antenna = '3,VA04' 
     timerange   =         ''   #   time range: ''==>all 
     correlation =         ''   #   correlations: default = '' 
     scan        =         ''   #   scan numbers: Not yet implemented
     feed        =         ''   #   multi-feed numbers: Not yet implemented
     array       =         ''   #   array numbers: Not yet implemented
     uvrange     =         ''   #   uv range''==>all; uvrange = '0~100kl' (default unit=meters)
\end{verbatim}
\normalsize
These are described in \S~\ref{section:io.selection}.

Averaging is controlled with the set of parameters
\small
\begin{verbatim}
averagemode      =   'vector'   #  Select averaging type: vector, scalar
     timebin     =        '0'   #  Length of time-interval in seconds to average
     crossscans  =      False   #  Have time averaging cross scan boundaries?
     crossbls    =      False   #  have averaging cross over baselines?
     crossarrays =      False   #  have averaging cross over arrays?
     stackspw    =      False   #  stack multiple spw on top of each other?
     width       =        '1'   #  Number of channels to average
\end{verbatim}
\normalsize
See \S~\ref{section:edit.plot.plotxy.average} below for more on averaging.

You can extend the flagging beyond the data cell plotted:
\small
\begin{verbatim}
extendflag          =   True    #  Have flagging extend to other data points?
     extendcorr     =     ''    #  flagging correlation extension type
     extendchan     =     ''    #  flagging channel extension type
     extendspw      =     ''    #  flagging spectral window extension type
     extendant      =     ''    #  flagging antenna extension type
     extendtime     =     ''    #  flagging time extension type
\end{verbatim}
\normalsize
See \S~\ref{section:edit.plot.plotxy.extend} below for more on flag
extension.

The {\tt restfreq} parameter can be set to a transition or frequency:
\small
\begin{verbatim}
restfreq            =    'HI'   #  a frequency quanta or transition name. see help for options
     frame          =  'LSRK'   #  frequency frame for spectral axis. see help for options
     doppler        = 'RADIO'   #  doppler mode. see help for options
\end{verbatim}
\normalsize
See \S~\ref{section:edit.plot.plotxy.restfreq} below for more on setting rest
frequencies and frames.

Setting {\tt selectplot=True} will open up a set of plotting control
sub-parameters.  
These are described in \S~\ref{section:edit.plot.plotxy.select} below.

The {\tt interactive} and {\tt figfile} parameters allow
non-interactive production of hardcopy plots.  See
\S~\ref{section:edit.plot.plotxy.print} for more details on saving plots to disk.

The {\tt iteration}, {\tt overplot}, {\tt plotrange}, 
{\tt plotsymbol}, {\tt showflags} and {\tt subplot}
parameters deserve extra explanation, and are described below.

For example:
\small
\begin{verbatim}
plotxy(vis='jupiter6cm.ms',                # jupiter 6cm dataset
       xaxis='uvdist',                     # plot uv-distance on x-axis
       yaxis='amp',                        # plot amplitude on y-axis
       field='JUPITER',                    # plot only JUPITER
       selectdata=True,                    # open data selection
       correlation='RR,LL',                #   plot RR and LL correlations
       selectplot=True,                    # open plot controls
       title = 'Jupiter 6cm uncalibrated') #   give it a title 
\end{verbatim}
\normalsize
The plotter resulting from these settings is shown in figure \ref{fig:matplotlib}.  

{\bf ALERT:} The {\tt plotxy} task still has a number of issues.
The averaging has been greatly speeded up in this release, but there
are cases where the plots will be made incorrectly.  In particular,
there are problems plotting multiple {\tt spw} at the same time.
There are sometimes also cases where data that you have flagged in 
{\tt plotxy} from averaged data is done so incorrectly.  This task is
under active developement for the next cycle to fix these remaining 
problems, so users should be aware of this.

{\bf ALERT:} Another know problem with ({\tt plotxy}) is that it
fails if the path to your working directory contains spaces in
its name, e.g. {\tt /users/smyers/MyTest/} is fine, but 
{\tt /users/smyers/My\ Test/} is not!

% There are a number of things to keep in mind and to be aware of
% in order to make your plotting and editing go smoother:

% The {\tt field} selection is a minimum match on a
% space-separated list of names. You can use the {\tt
% selectfield(vis=filename, minstring='string')} to test what field
% names and indices you are matching. Similarly, {\tt
% selectantenna(vis=filename, minstring='antname')} will also give you
% this information for antenna names.

%%%%%%%%%%%%%%%%%%%%%%%%%%%%%%%%%%%%%%%%%%%%%%%%%%%%%%%%%%%%%%%%%
\subsubsection{GUI Plot Control}
\label{section:edit.plot.plotxy.control}

You can use the various buttons on the {\tt plotxy} GUI to control
its operation -- in particular, to determine flagging and unflagging
behaviors.

There is a standard row of buttons at the bottom.  These include
(left to right):
\begin{itemize}
\item {\bf Home} --- The ``house'' button (1st on left) returns to
  the original zoom level.
\item {\bf Step} --- The left and right arrow buttons (2nd and 3rd
  from left) step through the zoom settings you've visited.
\item {\bf Pan} --- The ``four-arrow button'' (4th from left) lets you pan
  in zoomed plot.
\item {\bf Zoom} --- The most useful is the ``magnifying glass'' (5th
  from the left) which lets you draw a box and zoom in on the plot.  
\item {\bf Panels} --- The ``window-thingy'' button (second from
  right) brings up a menu to adjust the panel placement in the plot.
\item {\bf Save} -- The ``disk'' button (last on right) saves a
  {\tt .png} copy of the plot to a generically named file on disk.
\end{itemize}

In a row above these, there are a set of other buttons (left to right):
\begin{itemize}
\item {\bf Mark Region} --- If depressed lets you draw rectangles to
  mark points in the panels.  This is done by left-clicking and
  dragging the mouse.  You can Mark multiple boxes before doing
  something.  Clicking the button again will un-depress it and forget
  the regions.  ESC will remove the last region marked.
\item {\bf Flag} --- Click this to Flag the points in a marked region.
\item {\bf Unflag} --- Click this to Unflag any flagged point that
  would be in that region (even if invisible).
\item {\bf Locate} --- Print out some information to the logger on
  points in the marked regions.  
\item {\bf Next} --- Step to the next plot in an iteration.
\item {\bf Quit} --- Exit {\tt plotcal}, clear the window and detach from the MS.
\end{itemize}

These buttons are shared with the {\tt plotcal} tool.

%%%%%%%%%%%%%%%%%%%%%%%%%%%%%%%%%%%%%%%%%%%%%%%%%%%%%%%%%%%%%%%%%
\subsubsection{The {\tt selectplot} Parameters}
\label{section:edit.plot.plotxy.select}

These parameters work in concert with the native matplotlib
functionality to enable flexible representations of data displays. 

Setting {\tt selectplot=True} will open up a set of plotting control
sub-parameters:
\small
\begin{verbatim}
selectplot       =       True   #  Select additional plotting options (e.g, fontsize, title,etc)
     markersize  =        5.0   #  Size of plotted marks
     linewidth   =        1.0   #  Width of plotted lines
     skipnrows   =          1   #  Plot every nth point
     newplot     =      False   #  Replace the last plot or not when overplotting
     clearpanel  =     'Auto'   #  Specify if old plots are cleared or not
     title       =         ''   #  Plot title (above plot)
     xlabels     =         ''   #  Label for x-axis
     ylabels     =         ''   #  Label for y-axis
     fontsize    =       10.0   #  Font size for labels
     windowsize  =        5.0   #  Window size: not yet implemented
\end{verbatim}
\normalsize

\begin{wrapfigure}{r}{2.5in}
  \begin{boxedminipage}{2.5in}
     \centerline{\bf Inside the Toolkit:}
        For even more functionality, you can access the 
        {\tt pl} tool directly using Pylab functions that 
        allow one to annotate, alter, or add
        to any plot displayed in the {\tt matplotlib} plotter 
        (e.g. {\tt plotxy}).
  \end{boxedminipage}
\end{wrapfigure}

The {\tt markersize} parameter will change the size of the plot
symbols.  Increasing it will help legibility when doing screen shots.
Decreasing it can help in congested plots.  The {\tt linewidth}
parameter will do similar things to the lines.

The {\tt skipnrows} parameter, if set to an integer {\tt n} greater than 1,
will allow only every {\tt n}th point to be plotted.  It does this,
as the name suggests, by skipping over whole rows of the MS, so beware
(channels are all within the same row for a given {\tt spw}).  Be
careful flagging on data where you have skipped points!  Note
that you can also reduce the number of points plotted via averaging
(\S~\ref{section:edit.plot.plotxy.average}) or channel striding in the 
{\tt spw} specification (\S~\ref{section:io.selection.spw}).

The {\tt newplot} toggle lets you choose whether or not the
last layer plotted is replaced when {\tt overplot=True}, or whether
a new layer is added.

The {\tt clearpanel} parameter turns on/off the clearing of plot panels
that lie under the current panel layer being plotted. The options are:
{\tt 'none'} (clear nothing), {\tt 'auto'} (automatically clear the
plotting area), {\tt 'current'} (clear the current plot area only), 
and {\tt 'all'} (clear the whole plot panel).

The {\tt title}, {\tt xlabels}, and {\tt ylabels} parameters can
be used to change the plot title and axes labels.

The {\tt fontsize} parameter is useful in order to enlarge the label
fonts so as to be visible when making plots for screen capture, or
just to improve legibility.  Shrinking can help if you have lots of
panels on the plot also.

The {\tt windowsize} parameter is supposed to allow adjustments on
the window size. {\bf ALERT:} This currently does nothing,
unless you set it below 1.0, in which case it will produce an 
error.

%%%%%%%
\subsubsection{ The {\tt iteration} parameter}
\label{section:edit.plot.plotxy.iter}

There are currently four iteration options available:
{\tt 'field'}, {\tt 'antenna'}, and {\tt 'baseline'}.  
If one of these options
is chosen, the data will be split into separate plot displays for each
value of the iteration axis (e.g., for the VLA, the 'antenna' option
will get you 27 displays, one for each antenna).  

An example use of iteration:
\small
\begin{verbatim}
  # choose channel averaging, every 5 channels
  plotxy('n5921.ms','channel',subplot=221,iteration='antenna',width='5')
\end{verbatim}
\normalsize
The results of this are shown in Figure~\ref{fig:plotiter}.  Note
that this example combines the use of {\tt width}, {\tt iteration}
and {\tt subplot}.

% \begin{figure}[h!]
% \gname{msplot_iteration}{6.5}
% \caption{\label{fig:msplotiteration} The {\tt plotxy} iteration plot: The first
%   set of plots from the example in \S~\ref{section:edit.plot.plotxy.iter}. 
%   Each time you press the {\bf Next} button, you
%   get the next series of plots.} 
% \hrulefill
% \end{figure}

\begin{figure}[h!]
\begin{center}
\pngname{plotxy_iter}{6}
\caption{\label{fig:plotiter} The {\tt plotxy} iteration plot.
  The first set of plots from the example in
  \S~\ref{section:edit.plot.plotxy.iter} with {\tt iteration='antenna'}.
  Each time you press the {\bf Next} button, you
  get the next series of plots.} 
\hrulefill
\end{center}
\end{figure}

{\bf NOTE:} If you use {\tt iteration='antenna'} or {\tt 'baseline'},
be aware if you have set {\tt antenna} selection.  You can also
control whether you see auto-correlations or not using the appropriate
syntax, e.g. {\tt antenna='*\&\&*'} or {\tt antenna='*\&\&\&'}
(\S~\ref{section:io.selection.selectdata.antenna}).

%%%%%%%
\subsubsection{ The {\tt overplot} parameter}
\label{section:edit.plot.plotxy.overplot}

The {\tt overplot} parameter toggles whether the current plot will
be overlaid on the previous plot or subpanel (via the {\tt subplot}
setting, \S~{section:edit.plot.plotxy.subplot}) or will overwrite it.
The default is {\tt False} and the new plot will replace the old.

The {\tt overplot} parameter interacts with the {\tt newplot}
sub-parameter (see \S~\ref{section:edit.plot.plotxy.select}).

See \S~\ref{section:edit.plot.plotxy.showflags} for an example using 
{\tt overplot}. 

%%%%%%%
\subsubsection{ The {\tt plotrange} parameter}
\label{section:edit.plot.plotxy.plotrange}

The {\tt plotrange} parameter can be used to specify the size of the
plot.  The format is {\tt [xmin, xmax, ymin, ymax]}.  The units are
those on the plot.  For example,
\small
\begin{verbatim}
   plotrange = [-20,100,15,30]
\end{verbatim}
\normalsize
Note that if {\tt xmin=xmax} and/or {\tt ymin=ymax}, then the values
will be ignored and a best guess will be made to auto-range that axis.
{\tt ALERT:} Unfortunately, the units for the time axis must be
in Julian Days, which are the plotted values.

%%%%%%%
\subsubsection{ The {\tt plotsymbol} parameter}
\label{section:edit.plot.plotxy.symb}

The {\tt plotsymbol} parameter defines both the line or
symbol for the data being drawn as well as the color; from the
matplotlib online documentation (e.g., type {\tt pl.plot?} for help):

\small
\begin{verbatim}
    The following line styles are supported:
        -     : solid line
        --    : dashed line
        -.    : dash-dot line
        :     : dotted line
        .     : points
        ,     : pixels
        o     : circle symbols
        ^     : triangle up symbols
        v     : triangle down symbols
        <     : triangle left symbols
        >     : triangle right symbols
        s     : square symbols
        +     : plus symbols
        x     : cross symbols
        D     : diamond symbols
        d     : thin diamond symbols
        1     : tripod down symbols
        2     : tripod up symbols
        3     : tripod left symbols
        4     : tripod right symbols
        h     : hexagon symbols
        H     : rotated hexagon symbols
        p     : pentagon symbols
        |     : vertical line symbols
        _     : horizontal line symbols
        steps : use gnuplot style 'steps' # kwarg only
    The following color abbreviations are supported
        b  : blue
        g  : green
        r  : red
        c  : cyan
        m  : magenta
        y  : yellow
        k  : black
        w  : white
    In addition, you can specify colors in many weird and
    wonderful ways, including full names 'green', hex strings
    '#008000', RGB or RGBA tuples (0,1,0,1) or grayscale
    intensities as a string '0.8'.
    Line styles and colors are combined in a single format string, as in
    'bo' for blue circles.
\end{verbatim}
\normalsize

%%%%%%
\subsubsection{ The {\tt showflags} parameter}
\label{section:edit.plot.plotxy.showflags}

The {\tt showflags} parameter determines whether {\em only} unflagged
data ({\tt showflags=False}) or flagged ({\tt showflags=True}) data is
plotted by this execution.  The default is {\tt False} and will show
only unflagged ``good'' data.

Note that if you want to plot both unflagged and flagged data, in
different colors, then you need to run {\tt plotxy} twice using
{\tt overplot} (see \S~\ref{section:edit.plot.plotxy.overplot}) 
the second time, e.g.
\small
\begin{verbatim}
> plotxy(vis="myfile", xaxis='uvdist', yaxis='amp' )
> plotxy(vis="myfile", xaxis='uvdist', yaxis='amp', overplot=True, showflags=True )
\end{verbatim}
\normalsize

%%%%%%
\subsubsection{ The {\tt subplot} parameter }
\label{section:edit.plot.plotxy.subplot}

The {\tt subplot} parameter takes three
numbers. The first is the number of y panels (stacking vertically),
the second is the number of xpanels (stacking horizontally) and the
third is the number of the panel you want to draw into. For example,
{\tt subplot=212} would draw into the lower of two
panels stacked vertically in the figure.

An example use of subplot capability is shown in
Fig~\ref{fig:multiplot}.  These were drawn with the commands (for the
top, bottom left, and bottom right panels respectively):
\small
\begin{verbatim}
plotxy('n5921.ms','channel',     # plot channels for the n5921.ms data set
       field='0',                  # plot only first field
       datacolumn='corrected',     # plot corrected data
       plotcolor='',               # over-ride default plot color
       plotsymbol='go',            # use green circles
       subplot=211)                # plot to the top of two panels

plotxy('n5921.ms','x',           # plot antennas for n5921.ms data set
       field='0',                  # plot only first field
       datacolumn='corrected',     # plot corrected data
       subplot=223,                # plot to 3rd panel (lower left) in 2x2 grid
       plotcolor='',               # over-ride default plot color
       plotsymbol='r.')            # red dots

plotxy('n5921.ms','u','v',       # plot uv-coverage for n5921.ms data set
       field='0',                  # plot only first field
       datacolumn='corrected',     # plot corrected data
       subplot=224,                # plot to the lower right in a 2x2 grid
       plotcolor='',               # over-ride default plot color
       plotsymbol='b,')            # blue, somewhat larger dots
                                   # NOTE: You can change the gridding
                                   # and panel size by manipulating
                                   # the ny x nx grid.

\end{verbatim}
\normalsize

\begin{figure}[h!]
\begin{center}
%\gname{n5921_multiplot}{6}
\pngname{ngc5921_multiplot}{5.5}
\caption{\label{fig:multiplot} Multi-panel display of visibility
  versus channel ({\bf top}), antenna array configuration ({\bf bottom left})
  and the resulting uv coverage ({\bf bottom right}). The commands to
  make these three panels respectively are: 
  1) {\tt plotxy('ngc5921.ms', xaxis='channel',
    datacolumn='data', field='0', subplot=211, plotcolor='',
    plotsymbol='go')}
  2) {\tt plotxy('ngc5921.ms', xaxis='x', field='0', subplot=223, plotsymbol='r.')}, 
  3) {\tt plotxy('ngc5921.ms', xaxis='u', yaxis='v', field='0',
    subplot=224, plotsymbol='b,',figfile='ngc5921\_multiplot.png')}.
  }
\hrulefill
\end{center}
\end{figure}

See also \S~\ref{section:edit.plot.plotxy.iter} above, and
Figure~\ref{fig:plotiter} for an example of channel 
averaging using {\tt iteration} and {\tt subplot}.
 
%%%%%%%%%%%%%%%%%%%%%%%%%%%%%%%%%%%%%%%%%%%%%%%%%%%%%%%%%%%%%%%%%
\subsubsection{Averaging in {\tt plotxy}}
\label{section:edit.plot.plotxy.average}

The averaging parameters and sub-parameters are:
\small
\begin{verbatim}
averagemode      = 'vector' #  Select averaging type: vector, scalar
     timebin     =      '0' #  length of time in seconds to average, default='0', or: 'all'
     crossscans  =    False #  have time averaging cross over scans?
     crossbls    =    False #  have averaging cross over baselines?
     crossarrays =    False #  have averaging cross over arrays?
     stackspw    =    False #  stack multiple spw on top of each other?
     width       =      '1' #  number of channels to average, default: '1', or: 'all', 'allspw'
\end{verbatim}
\normalsize

The choice of {\tt averagemode} controls how the amplitudes are calculated
in the average.  The default mode is {\tt 'vector'}, where the complex
average is formed by averaging the real and imaginary parts of the
relevant visibilities.  If {\tt 'scalar'} is chosen, then the
amplitude of the average is formed by a scalar average of the
individual visibility amplitudes.

Time averaging is effected by setting the {\tt timebin} parameter to a
value larger than the integration time.  Currently, {\tt timebin}
takes a string containing the averaging time in seconds, e.g.
\small
\begin{verbatim}
   timebin = '60.0'
\end{verbatim}
\normalsize
to plot one-minute averages.

Channel averaging is invoked by setting {\tt width} to a value greater
than 1.  Currently, the averaging {\tt width} is given as a number of
channels.

By default, the averaging will not cross {\tt scan} boundaries (as set
in the import process).  However, if {\tt crossscans=True}, then
averaging will cross scans.  

Note that data taken in different sub-arrays are never averaged
together. Likewise, there is no way to plot data averaged over {\tt field}.

%%%%%%%%%%%%%%%%%%%%%%%%%%%%%%%%%%%%%%%%%%%%%%%%%%%%%%%%%%%%%%%%%
\subsubsection{Interactive Flagging in {\tt plotxy}}
\label{section:edit.plot.plotxy.flag}

\begin{wrapfigure}{r}{2.5in}
  \begin{boxedminipage}{2.5in}
     \centerline{\bf Hint!}
     In the plotting environments such as {\tt plotxy}, the
     {\tt ESC} key can be used to remove the last region box
     drawn. 
  \end{boxedminipage}
\end{wrapfigure}

Interactive flagging, on the principle of ``see it --- flag it'', is
possible on the X-Y display of the data plotted by {\tt plotxy}.  The user can
use the cursor to mark one or more regions, and then flag, unflag, or list
the data that falls in these zones of the display.

There is a row of buttons below the plot in the window.  You can punch the
{\bf Mark Region} button (which will appear to depress), then mark a region
by left-clicking and dragging the mouse (each click and drag will mark an
additional region).  You can get rid of all your regions by clicking again
on the {\bf Mark Region} button (which will appear to un-depress), or you
can use the {\tt ESC} key to remove the last box you drew.  Once regions are
marked, you can then click on one of the other buttons to take action:
\begin{enumerate}
\item {\bf Flag} --- flag the points in the region(s),
\item {\bf Unflag} --- unflag flagged points in the region(s),
\item {\bf Locate} --- spew out a list of the points in the region(s) to 
   the logger (Warning: this could be a long list!).
\end{enumerate}
Whenever you click on a button, that action occurs without 
forcing a disk-write (unlike previous versions).  If you quit {\tt plotxy}
and re-enter, you will see your previous edits.

\begin{figure}[h!]
\begin{center}
%\gname{markflags}{3}
%\gname{markflags2}{3}
\pngname{ngc5921_markflag2}{2.75}
\pngname{ngc5921_markflag3}{2.75}
\caption{\label{fig:markflags} Plot of amplitude versus
uv distance, before (left) and after (right) flagging two marked regions.
The call was:
{\tt plotxy(vis='ngc5921.ms',xaxis='uvdist', field='1445*')}.
}
\hrulefill
\end{center}
\end{figure}

A table with the name {\tt <msname>.flagversions} (where
{\tt vis=<msname>}) will be created in the same directory if
it does not exist already.

It is recommended that you save important flagging stages using
the {\tt flagmanager} task (\S~\ref{section:edit.flagmanager}).

%%%%%%%%%%%%%%%%%%%%%%%%%%%%%%%%%%%%%%%%%%%%%%%%%%%%%%%%%%%%%%%%%
\subsubsection{Flag extension in {\tt plotxy}}
\label{section:edit.plot.plotxy.extend}

Flag extension is controlled using {\tt extendflag=T} and its sub-parameters:
\small
\begin{verbatim}
extendflag          =   True    #  Have flagging extend to other data points?
     extendcorr     =     ''    #  flagging correlation extension type
     extendchan     =     ''    #  flagging channel extension type
     extendspw      =     ''    #  flagging spectral window extension type
     extendant      =     ''    #  flagging antenna extension type
     extendtime     =     ''    #  flagging time extension type
\end{verbatim}
\normalsize
The use of {\tt extendflag} enables the user to plot a subset of the
data and extend the flagging to a wider set.

{\bf ALERT:} Using the {\tt extendflag} options will greatly
slow down the flagging in {\tt plotxy}.  You will see a long delay
after hitting the {\bf Flag} button, with lots of logger messages
as it goes through each flag.  Fixing this requires a refactoring
of {\tt plotxy} which is underway starting in Patch 4 development.

Setting {\tt extendchan='all'} will extend the flagging to other
channels in the same {\tt spw} as the displayed point.  For example,
if {\tt spw='0:0'} and channel 0 is displayed, then flagging will
extend to all channels in spw 0.

The {\tt extendcorr} sub-parameter will extend the flagging beyond the
correlations displayed.  If {\tt extendcorr='all'}, then all
correlations will be flagged, e.g.\ with RR displayed RR,RL,LR,LL will 
be flagged.  If {\tt extendcorr='half'}, then the extension will be
to those correlations in common with that show, e.g.\ with RR
displayed then RR,RL,LR will be flagged.

Setting {\tt extendspw='all'} will extend the flagging to all other
spw for the selection.  Using the same example as above, with
{\tt spw='0:0'} displayed, then channel 0 in ALL spw will be flagged.
Note that use of {\tt extendspw} could result in unintended behavior
if the spw have different numbers of channels, or if it is used in
conjunction with {\tt extendchan}.

{\bf WARNING:} use of the following options, particularly in
conjunction with other flag extensions, may lead to deletion of much
more data than desired.  Be careful!

Setting {\tt extendant='all'} will extend the flagging to all
baselines that have antennas in common with those displayed and
marked.  For example, if {\tt antenna='1\&2'} is shown, then ALL
baselines to BOTH antennas 1 and 2 will be flagged.  Currently, there
is no option to extend the flag to ONLY baselines to the first (or 
second) antenna in a displayed pair, so it is better to use
{\tt flagdata} to remove specific antennas.

Setting {\tt extendtime='all'} will extend the flagging to all times 
matching the other selection or extension for the data in the marked
region.  

%%%%%%%%%%%%%%%%%%%%%%%%%%%%%%%%%%%%%%%%%%%%%%%%%%%%%%%%%%%%%%%%%
\subsubsection{Setting rest frequencies in {\tt plotxy}}
\label{section:edit.plot.plotxy.restfreq}

The {\tt restfreq} parameter can be set to a transition or frequency
and expands to allow setting of frame information.  For example,
\small
\begin{verbatim}
restfreq            =    'HI'   #  a frequency quanta or transition name. see help for options
     frame          =  'LSRK'   #  frequency frame for spectral axis. see help for options
     doppler        = 'RADIO'   #  doppler mode. see help for options
\end{verbatim}
\normalsize
Examples of transitions include:
\small
\begin{verbatim}
   restfreq='1420405751.786Hz'  #  21cm HI frequency
   restfreq='HI'                #  21cm HI transition name
   restfreq='115.2712GHz'       #  CO 1-0 line frequency
\end{verbatim}
\normalsize
For a list of known lines in the CASA {\tt measures} system, use the
toolkit command {\tt me.linelist()}.  For example:
\small
\begin{verbatim}
CASA <14>: me.linelist()
  Out[14]: 'C109A CI CII166A DI H107A H110A H138B H166A H240A H272A H2CO HE110A HE138B HI OH1612 OH1665 OH1667 OH1720'
\end{verbatim}
\normalsize
{\bf ALERT:} The list of known lines in CASA is currently very
restricted, and will be increased in upcoming releases (to include lines
in ALMA bands for example).

You can use the {\tt me.spectralline} tool method to turn transition names into
frequencies 
\small
\begin{verbatim}
CASA <16>: me.spectralline('HI')
  Out[17]: 
{'m0': {'unit': 'Hz', 'value': 1420405751.786},
 'refer': 'REST',
 'type': 'frequency'}
\end{verbatim}
\normalsize
(not necessary for this task, but possibly useful).

The {\tt frame} sub-parameter sets the frequency frame.  The allowed
options can be listed using the {\tt me.listcodes} method on the
{\tt me.frequency()} method, e.g.
\small
\begin{verbatim}
CASA <17>: me.listcodes(me.frequency())
  Out[17]: 
{'extra': array([], 
      dtype='|S1'),
 'normal': array(['REST', 'LSRK', 'LSRD', 'BARY', 'GEO', 'TOPO', 'GALACTO', 'LGROUP',
       'CMB'], 
      dtype='|S8')}
\end{verbatim}
\normalsize

The {\tt doppler} sub-parameter likewise sets the Doppler system.  The
allowed codes can be listed using the {\tt me.listcodes} method on the
{\tt me.doppler()} method,
\small
\begin{verbatim}
CASA <18>: me.listcodes(me.doppler())
  Out[18]: 
{'extra': array([], 
      dtype='|S1'),
 'normal': array(['RADIO', 'Z', 'RATIO', 'BETA', 'GAMMA', 'OPTICAL', 'TRUE',
       'RELATIVISTIC'], 
      dtype='|S13')}
\end{verbatim}
\normalsize
For most cases the {\tt 'RADIO''} Doppler system is appropriate, but
be aware of differences.

For more information on frequency frames and spectral coordinate
systems, see the paper by Greisen et al. (A\&A, 446, 747, 2006)
\footnote{Also at \url{http://www.aoc.nrao.edu/~egreisen/scs.ps}}.

%%%%%%%%%%%%%%%%%%%%%%%%%%%%%%%%%%%%%%%%%%%%%%%%%%%%%%%%%%%%%%%%%
\subsubsection{Printing from {\tt plotxy}}
\label{section:edit.plot.plotxy.print}

There are two ways to get hardcopy plots in {\tt plotxy}.  

The first is to use the ``disk save'' icon from the interactive plot GUI to 
print the current plot.  This will bring up a sub-menu GUI that will
allow you to choose the filename and format.  The allowed formats are {\tt .png} (PNG), 
{\tt .eps} (EPS), and {\tt svg} (SVG).  If you give the filename with
a suffix ({\tt .png}, {\tt .eps}, or {\tt svg}) it will make a plot of
that type.  Otherwise it will put a suffix on depending on the format
chosen from the menu.

{\bf ALERT:} The plot files produced by the EPS option can be
large, and the SVG files can be very large.  The PNG is the smallest.

The second is to specify a {\tt figfile}.  You probably want to disable the
GUI using {\tt interactive=False} in this case.  The type of plot file
that is made will depend upon the filename suffix.  The allowed
choices are {\tt .png} (PNG), {\tt .eps} (EPS), and {\tt svg} (SVG).

This latter option is most useful from scripts.  For example,
\small
\begin{verbatim}
   default('plotxy')
   vis = 'ngc5921.ms'
   field = '2'
   spw = ''
   xaxis = 'uvdist'
   yaxis = 'amp'
   interactive=False
   figfile = 'ngc5921.uvplot.amp.png'
   plotxy()
\end{verbatim}
\normalsize
will plot amplitude versus uv-distance in PNG format.  No {\tt plotxy}
GUI will appear.

{\bf ALERT:} if
you use this option to print to {\tt figfile} with an {\tt iteration}
set, you will only get the first plot.

%%%%%%%%%%%%%%%%%%%%%%%%%%%%%%%%%%%%%%%%%%%%%%%%%%%%%%%%%%%%%%%%%
\subsubsection{Exiting {\tt plotxy}}
\label{section:edit.plot.plotxy.exit}

You can use the {\bf Quit} button to clear the plot from the
window and detach from the MS.  You can also dismiss the 
window by killing it with the X on the frame, which will also
detach the MS.

You can also just leave it alone.  The plotter pretty much keeps running
in the background even when it looks like it's done!  You can
keep doing stuff in the plotter window, which is where the
{\tt overplot} parameter comes in.  Note that the {\tt plotcal}
task (\S~\ref{section:cal.tables.plotcal}) will use the same window, and
can also overplot on the same panel.

If you leave {\tt plotxy} running, beware of (for instance)
deleting or writing over the MS without stopping.  It may work
from a memory version of the MS or crash.

%%%%%%%%%%%%%%%%%%%%%%%%%%%%%%%%%%%%%%%%%%%%%%%%%%%%%%%%%%%%%%%%%
\subsubsection{Example session using {\tt plotxy}}
\label{section:edit.plot.plotxy.example}

The following is an example of interactive plotting and flagging
using {\tt plotxy} on the Jupiter 6cm continuum VLA dataset.
This is extracted from the script {\tt jupiter6cm\_usecase.py}
available in the script area.

This assumes that the MS {\tt jupiter6cm.usecase.ms} is
on disk with {\tt flagautocorr} already run.

\small
\begin{verbatim}
default('plotxy')

vis = 'jupiter6cm.usecase.ms'

# The fields we are interested in: 1331+305,JUPITER,0137+331
selectdata = True

# First we do the primary calibrator
field = '1331+305'

# Plot only the RR and LL for now
correlation = 'RR LL'

# Plot amplitude vs. uvdist
xaxis = 'uvdist'
yaxis = 'amp'
multicolor = 'both'

# The easiest thing is to iterate over antennas
iteration = 'antenna'

plotxy()

# You'll see lots of low points as you step through RR LL RL LR
# A basic clip at 0.75 for RR LL and 0.055 for RL LR will work
# If you want to do this interactively, set
iteration = ''

plotxy()

# You can also use flagdata to do this non-interactively
# (see below)

# Now look at the cross-polar products
correlation = 'RL LR'

plotxy()

#---------------------------------------------------------------------
# Now do calibrater 0137+331
field = '0137+331'
correlation = 'RR LL'
xaxis = 'uvdist'
spw = ''
iteration = ''
antenna = ''

plotxy()

# You'll see a bunch of bad data along the bottom near zero amp
# Draw a box around some of it and use Locate
# Looks like much of it is Antenna 9 (ID=8) in spw=1

xaxis = 'time'
spw = '1'
correlation = ''

# Note that the strings like antenna='9' first try to match the 
# NAME which we see in listobs was the number '9' for ID=8.
# So be careful here (why naming antennas as numbers is bad).
antenna = '9'

plotxy()

# YES! the last 4 scans are bad.  Box 'em and flag.

# Go back and clean up
xaxis = 'uvdist'
spw = ''
antenna = ''
correlation = 'RR LL'

plotxy()

# Box up the bad low points (basically a clip below 0.52) and flag

# Note that RL,LR are too weak to clip on.

#---------------------------------------------------------------------
# Finally, do JUPITER
field = 'JUPITER'
correlation = ''
iteration = ''
xaxis = 'time'

plotxy()

# Here you will see that the final scan at 22:00:00 UT is bad
# Draw a box around it and flag it!

# Now look at whats left
correlation = 'RR LL'
xaxis = 'uvdist'
spw = '1'
antenna = ''
iteration = 'antenna'

plotxy()

# As you step through, you will see that Antenna 9 (ID=8) is often 
# bad in this spw. If you box and do Locate (or remember from
# 0137+331) its probably a bad time.

# The easiset way to kill it:

antenna = '9'
iteration = ''
xaxis = 'time'
correlation = ''

plotxy()

# Draw a box around all points in the last bad scans and flag 'em!

# Now clean up the rest
xaxis = 'uvdist'
correlation = 'RR LL'
antenna = ''
spw = ''

# You will be drawing many tiny boxes, so remember you can
# use the ESC key to get rid of the most recent box if you
# make a mistake.

plotxy()

# Note that the end result is we've flagged lots of points
# in RR and LL.  We will rely upon imager to ignore the
# RL LR for points with RR LL flagged!

\end{verbatim}
\normalsize

%%%%%%%%%%%%%%%%%%%%%%%%%%%%%%%%%%%%%%%%%%%%%%%%%%%%%%%%%%%%%%%%%
\subsection{Plotting antenna positions using {\tt plotants}}
\label{section:edit.plot.plotants}

This task is a simple plotting interface (to the {\tt plotxy}
functionality) to produce plots of the antenna positions (taken from
the {\tt ANTENNA} sub-table of the MS).

The inputs to {\tt plotants} are:
\small
\begin{verbatim}
#  plotants :: Plot the antenna distribution in the local reference frame:
vis       =         ''   #  Name of input visibility file (MS)
figfile   =         ''   #  Save the plotted figure to this file
async     =      False   #  
\end{verbatim}
\normalsize

%%%%%%%%%%%%%%%%%%%%%%%%%%%%%%%%%%%%%%%%%%%%%%%%%%%%%%%%%%%%%%%%%
%%%%%%%%%%%%%%%%%%%%%%%%%%%%%%%%%%%%%%%%%%%%%%%%%%%%%%%%%%%%%%%%%
\section{Non-Interactive Flagging using {\tt flagdata}}
\label{section:edit.flagdata}

Task {\tt flagdata} will flag the visibility data set based on the
specified data selections, most of the information coming from a run
of the {\tt listobs} task (with/without {\tt verbose=True}). Currently you can
select based on any combination of: 

\begin{itemize}
   \item antennas ({\tt antenna})
   \item baselines ({\tt antenna})
   \item spectral windows and channels ({\tt spw})
   \item correlation types ({\tt correlation})
   \item field ids or names ({\tt field})
   \item uv-ranges ({\tt uvrange})
   \item times ({\tt timerange}) or scan numbers ({\tt scan})
   \item antenna arrays ({\tt array})
\end{itemize}

and choose to flag, unflag, clip ({\tt setclip} and sub-parameters), and
remove the first part of each scan ({\tt setquack}) and/or the 
autocorrelations ({\tt autocorr}).

The inputs to {\tt flagdata} are:
\small
\begin{verbatim}
#  flagdata :: Flag/Clip data based on selections:

vis               =         ''   #  Name of input visibility file
mode              = 'manualflag' #  Mode (manualflag,shadow,quack,summary,autoflag,rfi)
     autocorr     =      False   #  Flag autocorrelations
     unflag       =      False   #  Unflag the data specified
     clipexpr     =   'ABS RR'   #  Expression to clip on
     clipminmax   =         []   #  Range to use for clipping
     clipcolumn   =     'DATA'   #  Data column to use for clipping
     clipoutside  =       True   #  Clip outside the range, or within it
     channelavg   =      False   #  Average over channels

spw               =         ''   #  spectral-window/frequency/channel
field             =         ''   #  Field names or field index numbers: ''==>all, field='0~2,3C286'
selectdata        =       True   #  More data selection parameters (antenna, timerange etc)
     antenna      =         ''   #  antenna/baselines: ''==>all, antenna = '3,VA04'
     timerange    =         ''   #  time range: ''==>all, timerange='09:14:0~09:54:0'
     correlation  =         ''   #  Select data based on correlation
     scan         =         ''   #  scan numbers: ''==>all
     feed         =         ''   #  multi-feed numbers: Not yet implemented
     array        =         ''   #  (sub)array numbers: ''==>all
     uvrange      =         ''   #  uv range: ''==>all; uvrange = '0~100klambda', default units=meters

async             =      False   #  If true the taskname must be started using flagdata(...)
\end{verbatim}
\normalsize

The default flagging {\tt mode} is {\tt 'manualflag'}.  See 
\S~\ref{section:edit.flagdata.clip} more more on this option.

The {\tt mode='summary'} will print out a summary of the current
state of flagging into the {\tt logger}.

The {\tt mode='quack'} will allow dropping of integrations from the
beginning of scans.  See \S~\ref{section:edit.flagdata.quack} for
details.

The {\tt mode='shadow'} option will allow shadowed data to be flagged,
e.g.\ if it has not already during filling 
(\S~\ref{section:edit.flagdata.shadow}).

There are also ``autoflagging'' modes {\tt'autoflag'} and {\tt'rfi'} 
available for testing (\S~\ref{section:edit.flagdata.autoflag}).

Note that often you want to apply many different flagging operations
in a single pass through the data.  For this, we have implemented
a ``list'' mode for the {\tt 'manualflag'} selections where
several flagdata task invocations can be combined into a single
{\tt flagdata} run by giving lists of parameters. 
This is possible for the modes {\tt 'manualflag'} and {\tt 'quack'}, only.
    
For example, the following three flagdata runs:
\small
\begin{verbatim}
   flagdata(vis='my.ms', mode='manualflag', selectdata=True, field='3', autocorr=True)
   flagdata(vis='my.ms', mode='manualflag', selectdata=True, field='3', timerange = '6:0:0~6:23:00')
   flagdata(vis='my.ms', mode='manualflag', selectdata=True, field='3', scan='0', spw='0:60;62;64')
\end{verbatim}
\normalsize
can be combined into a single run by:
\small
\begin{verbatim}
   vis = 'my.ms'
   mode = 'manualflag'
   selectdata = True
   
   field     = '3'
   spw       = [ ''   , ''            , '0:60;62:64' ]
   autocorr  = [ True , False         , False        ]
   timerange = [ ''   , '6:0:0~6:23:0', ''           ]
   scan      = [ ''   , ''            , '0'          ]
   
   flagdata()
\end{verbatim}
\normalsize
Note that {\tt field='3'} is equivalent to {\tt field=['3','3','3']}.

%%%%%%%%%%%%%%%%%%%%%%%%%%%%%%%%%%%%%%%%%%%%%%%%%%%%%%%%%%%%%%%%%
-------------------------
\subsection{Flag Antenna/Channels}
\label{section:edit.flagdata.ant}

The following commands give the results shown in 
Figure\,\ref{fig:flagdata_antchan}:
\small
\begin{verbatim}
  default{'plotxy')
  plotxy('ngc5921.ms','channel',iteration='antenna',subplot=311)
  default('flagdata')
  flagdata(vis='ngc5921.ms',antenna='0',spw='0:10~15')
  default plotxy
  plotxy('ngc5921.ms','channel',iteration='antenna',subplot=311)
\end{verbatim}
\normalsize

\begin{figure}[h!]
\begin{center}
\pngname{msplot_channelants}{3}
\pngname{msplot_flagantchan}{3}
\caption{\label{fig:flagdata_antchan} {\tt flagdata}: Example showing before
  and after displays using a selection of one antenna and a range of
  channels. Note that each invocation of the flagdata task represents
  a cumulative selection, i.e., running antenna='0' will flag all
  data with antenna 0, while antenna='0', spw='0:10~15'
  will flag only those channels on antenna 0. }
\hrulefill
\end{center}
\end{figure}


%%%%%%%%%%%%%%%%%%%%%%%%%%%%%%%%%%%%%%%%%%%%%%%%%%%%%%%%%%%%%%%%%
\subsubsection{Manual flagging and clipping in {\tt flagdata}}
\label{section:edit.flagdata.clip}

For {\tt mode='manualflag''}, manual flagging and clipping is controlled by the
sub-parameters:
\small
\begin{verbatim}
mode              = 'manualflag' #  Mode (manualflag,shadow,quack,summary,autoflag,rfi)
     autocorr     =      False   #  Flag autocorrelations
     unflag       =      False   #  Unflag the data specified
     clipexpr     =   'ABS RR'   #  Expression to clip on
     clipminmax   =         []   #  Range to use for clipping
     clipcolumn   =     'DATA'   #  Data column to use for clipping
     clipoutside  =       True   #  Clip outside the range, or within it
     channelavg   =      False   #  Average over channels
\end{verbatim}
\normalsize

The following commands give the results shown in 
Figure\,\ref{fig:flagdata}:
\small
\begin{verbatim}
  plotxy('ngc5921.ms','uvdist')
  flagdata(vis='ngc5921.ms',clipexpr='LL',clipminmax=[0.0,1.6],clipoutside=True)
  plotxy('ngc5921.ms','uvdist')
\end{verbatim}
\normalsize

\begin{figure}[h!]
\begin{center}
\pngname{msplot_clipbefore}{3}
\pngname{msplot_clipafter}{3}
\caption{\label{fig:flagdata} {\tt flagdata}: Flagging example using the
  clip facility. }
\hrulefill
\end{center}
\end{figure}

The {\tt channelavg} toggle (new in Version 3.0.0) is now available
to (vector) average the data over all channels before doing the
clipping test.  This is most useful when flagging on phase stable or
corrected data (e.g.\ after {\tt applycal} and split to a new
dataset).

%%%%%%%%%%%%%%%%%%%%%%%%%%%%%%%%%%%%%%%%%%%%%%%%%%%%%%%%%%%%%%%%%
\subsubsection{Flagging the beginning of scans}
\label{section:edit.flagdata.quack}

You can use the {\tt mode='quack'} option to drop integrations from
the beginning of scans (as in the AIPS task {\tt QUACK}):
\small
\begin{verbatim}
mode              = 'quack'  # Mode (manualflag,shadow,quack,summary,autoflag,rfi)
   autocorr       =   False  # Flag autocorrelations
   unflag         =   False  # Unflag the data specified
   quackinterval  =     0.0  # Quack n seconds from quackmode scan boundary
   quackmode      =   'beg'  # 'beg' (start), 'endb' (end), 'end' (all but end), 'tail' (all but start)
   quackincrement =   False  #  Flag incrementally in time?
\end{verbatim}
\normalsize
Note that the time is measured from the first integration in the MS
for a given scan, and this is often already flagged by the online
system.

For example, assuming the integration time is 3.3 seconds (e.g. for
VLA), then
\small
\begin{verbatim}
   mode = 'quack'
   quackinterval = 14.0 
\end{verbatim}
\normalsize
will flag the first 4 integrations in every scan.

%%%%%%%%%%%%%%%%%%%%%%%%%%%%%%%%%%%%%%%%%%%%%%%%%%%%%%%%%%%%%%%%%
\subsubsection{Flagging shadowed data with mode {\tt 'shadow'} }
\label{section:edit.flagdata.shadow}

Visibilities where one antenna is obscured by another antenna
(typically at low elevations) can be flagged with the {\tt 'shadow'}
mode:
\small
\begin{verbatim}
mode          =   'shadow'  #  Mode (manualflag,shadow,quack,summary,autoflag,rfi)
   diameter   =       -1.0  #  Effective diameter (m) to use. -1 ==>antenna diameter
\end{verbatim}
\normalsize
Note that this can only flag data shadowed by antennas known in the MS
(in the same subarray for example), not by antennas not in the dataset.

%%%%%%%%%%%%%%%%%%%%%%%%%%%%%%%%%%%%%%%%%%%%%%%%%%%%%%%%%%%%%%%%%
\subsubsection{Autoflagging using modes {\tt 'autoflag'} and {\tt 'rfi'}}
\label{section:edit.flagdata.autoflag}

There are two ``autoflagging'' modes currently available in {\tt flagdata}.
{\bf ALERT:} These are still experimental and under development for
ALMA and EVLA commissioning.

The {\tt mode='autoflag'} option uses an older autoflagging algorithm
developed for {\tt aips++}:
\small
\begin{verbatim}
mode          = 'autoflag'  #  Mode (manualflag,shadow,quack,summary,autoflag,rfi)
   algorithm  =  'timemed'  #  Autoflag algorithm (timemed,freqmed)
   column     =     'DATA'  #  Data column to operate on.
   expr       =   'ABS RR'  #  Expression to flag on
   thr        =        5.0  #  Flagging threshold (n sigma)
   window     =         10  #  Sliding median filter width
\end{verbatim}
\normalsize
This algorithm passes median filters of a given width ({\tt window})
in time ({\tt timemed'}) or frequency ({\tt freqmed'}) over the data
and flagging outliers above a threshold ({\tt 'thr'}).

The {\tt mode='rfi'} option uses an newer autoflagging algorithm:
\small
\begin{verbatim}
mode               =    'rfi'  # Mode (manualflag,shadow,quack,summary,autoflag,rfi)
   clipcolumn      =   'DATA'  # Data column to use for clipping
   clipexpr        = 'ABS RR'  # Expression to clip on
   time_amp_cutoff =      4.0  # Flagging thresholds in units of sigma
   freq_amp_cutoff =      3.0  # Flagging thresholds in units of sigma
   freqlinefit     =    False  # Fit bandpass with a straight line (True), else piecewise polynomial
   auto_cross      =        1  # Flag on cross and auto-correlations (0=autocor only)
   num_time        =      400  # Number of time-steps in each chunk
   start_chan      =        1  # Channel range start (1 based)
   end_chan        =     2048  # Channel range end (1 based)
   bs_cutoff       =      0.0  # Baseline tolerance
   ant_cutoff      =      0.0  # Total autocorrelation amplitude threshold for a functional antenna
   flag_level      =        1  # 0=no extend; 1=also one timestep before+after; 2= also one channel
\end{verbatim}
\normalsize


%%%%%%%%%%%%%%%%%%%%%%%%%%%%%%%%%%%%%%%%%%%%%%%%%%%%%%%%%%%%%%%%%
\section{Browse the Data}
\label{section:edit.browse}

The {\tt browsetable} task is available for viewing data directly
(and handles all CASA tables, including Measurement Sets, calibration tables,
and images). This task brings up the CASA Qt
{\tt casabrowser}, which is a separate program.  You can launch this
from outside {\tt casapy}.  

The default inputs are:
\small
\begin{verbatim}
#  browsetable :: Browse a table (MS, calibration table, image)

tablename         =         ''  #   Name of input table
async             =      False  #  If true the taskname must be started using browsetable(...)

\end{verbatim}
\normalsize

Currently, its single input is the {\tt tablename}, so an example would
be:
\small
\begin{verbatim}
   browsetable('ngc5921.ms')
\end{verbatim}
\normalsize
For an MS such as this, it will come up with a browser of the 
{\tt MAIN} table (see Fig~\ref{fig:qcasabrowser1}).  
If you want to look at sub-tables, use the tab 
{\bf table keywords} along the left side to bring up a panel with the sub-tables
listed (Fig~\ref{fig:qcasabrowser2}), then choose (left-click) a table and
{\bf View:Details} to bring it up (Fig~\ref{fig:qcasabrowser3}).  
You can left-click on a cell in a table to view the
contents.

\begin{figure}[h!]
\begin{center}
%\gname{tablebrowser2}{6}
\pngname{qcasabrowser1}{6}
\caption{\label{fig:qcasabrowser1} {\tt browsetable}: The browser displays
  the main table within a frame. You can scroll
  through the data (x=columns of the {\tt MAIN} table, and y=the rows) or
  select a specific page or row as desired.  By default, 1000 rows of
  the table are loaded at a time, but you can step through the MS in batches.} 
\hrulefill
\end{center}
\end{figure}

\begin{figure}[h!]
\begin{center}
%\gname{tablebrowser3}{6}
\pngname{qcasabrowser2}{6}
\caption{\label{fig:qcasabrowser2} {\tt browsetable}: You can use the
  tab for {\tt Table Keywords} to look at other tables within an MS.
  You can then double-click on a table to view its contents.} 
\hrulefill
\end{center}
\end{figure}
 
\begin{figure}[h!]
\begin{center}
%\gname{tablebrowser4}{6}
\pngname{qcasabrowser3}{6}
\caption{\label{fig:qcasabrowser3} {\tt browsetable}: Viewing the 
{\tt SOURCE} table of the MS.}
\hrulefill
\end{center}
\end{figure}

Note that one useful feature is that you can Edit the table and its
contents.  Use the {\tt Edit table} choice from the {\bf Edit} menu,
or click on the {\bf Edit} button.  Be careful with this, and make
a backup copy of the table before editing!

Use the {\tt Close Tables and Exit} option from the {\bf Files} menu
to quit the {\tt casabrowser}.

There are alot of features in the {\tt casabrowser}
that are not fully documented here.  Feel free to explore the
capabilities such as plotting and sorting!

{\bf ALERT:} You are likely to find that the {\tt casabrowser}
needs to get a table lock before proceeding.  Use the {\tt clearstat}
command to clear the lock status in this case.

%%%%%%%%%%%%%%%%%%%%%%%%%%%%%%%%%%%%%%%%%%%%%%%%%%%%%%%%%%%%%%%%%
%%%%%%%%%%%%%%%%%%%%%%%%%%%%%%%%%%%%%%%%%%%%%%%%%%%%%%%%%%%%%%%%%
\section{Examples of Data Display and Flagging}
\label{section:edit.examples}

See the scripts provied in Appendix~\ref{chapter:scripts} for examples of
data examination and flagging.  In particular, we refer
the interested user to the demonstrations for:
\begin{itemize}
\item NGC5921 (VLA HI) --- a quick demo of basic CASA capabilities
      (\ref{section:scripts.ngc5921})
\item Jupiter (VLA 6cm continuum polarimetry) --- more extensive
      editing
      (\ref{section:scripts.jupiter})
\end{itemize}

%%%%%%%%%%%%%%%%%%%%%%%%%%%%%%%%%%%%%%%%%%%%%%%%%%%%%%%%%%%%%%%%%
%%%%%%%%%%%%%%%%%%%%%%%%%%%%%%%%%%%%%%%%%%%%%%%%%%%%%%%%%%%%%%%%%
